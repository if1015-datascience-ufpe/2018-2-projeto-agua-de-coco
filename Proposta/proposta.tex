\documentclass[review]{elsarticle}

\usepackage[brazilian]{babel}
\usepackage[utf8]{inputenc}
\usepackage[T1]{fontenc}

\usepackage{lineno,hyperref}
\modulolinenumbers[5]

%%%%%%%%%%%%%%%%%%%%%%%
%% Elsevier bibliography styles
%%%%%%%%%%%%%%%%%%%%%%%
%% To change the style, put a % in front of the second line of the current style and
%% remove the % from the second line of the style you would like to use.
%%%%%%%%%%%%%%%%%%%%%%%

%% Numbered
%\bibliographystyle{model1-num-names}

%% Numbered without titles
%\bibliographystyle{model1a-num-names}

%% Harvard
%\bibliographystyle{model2-names.bst}\biboptions{authoryear}

%% Vancouver numbered
%\usepackage{numcompress}\bibliographystyle{model3-num-names}

%% Vancouver name/year
%\usepackage{numcompress}\bibliographystyle{model4-names}\biboptions{authoryear}

%% APA style
%\bibliographystyle{model5-names}\biboptions{authoryear}

%% AMA style
%\usepackage{numcompress}\bibliographystyle{model6-num-names}

%% `Elsevier LaTeX' style
\bibliographystyle{elsarticle-num}
%%%%%%%%%%%%%%%%%%%%%%%

\begin{document}

\begin{frontmatter}

\title{Proposta de Projeto}

%% Authors
\author{Alexsandro Carvalho}
\ead{avsc@cin.ufpe.br}

\author{Jeffson Simões}
\ead{jcss3@cin.ufpe.br}

\begin{abstract}
Este texto propõe um projeto para a cadeira de Introdução à Ciência dos Dados do Centro de Informática - UFPE.
\end{abstract}

\begin{keyword}
data science\sep tourism
\end{keyword}

\end{frontmatter}

\linenumbers

\section{Introdução}
Nosso grupo decidiu analisar a influência dos diversos empreendimentos turísticos existentes no Brasil sobre a quantidade de turistas estrangeiros que visitam nosso país.

\section{Os Dados}
Os dados que pretendemos utilizar vêm do Portal Brasileiro de Dados Abertos\footnote{Portal Brasileiro de Dados Abertos: dados.gov.br}. Os datasets são: a quantidade de turistas de cada país que visitam cada estado do Brasil por mês\footnote{Turistas visitando o país: dados.gov.br/dataset/chegada-turistas} e as listas de pessoas físicas e jurídicas que atuam no setor do turismo\footnote{Dados do minstério do turismo: dados.gov.br/organization/ministerio-do-turismo-mtur}, incluindo:
\begin{itemize}
\item Restaurantes, cafeterias e bares
\item Organizadoras de eventos
\item Locadoras de veículos
\item Empreendimentos de entretenimento e lazer
\item Centros de convenções
\item Parques temáticos
\item Meios de hospedagem
\end{itemize}

\section{Análises Propostas}
Avaliar se existe uma relação entre os idiomas oferecidos nas hospedagens em cada estado e os países que mais os visitam. Para isso, serão plotados gráficos que exibirão a frequência de cada idioma nas hospedagens e nos países.

Avaliar a relação entre a quantidade de cada tipo de empreendimento e a chegada de turistas. Para isso, será usado um modelo de regressão.

\end{document}